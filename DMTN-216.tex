\documentclass[DM,authoryear,toc]{lsstdoc}
% lsstdoc documentation: https://lsst-texmf.lsst.io/lsstdoc.html
\input{meta}

% Package imports go here.
\usepackage{amsmath}
\usepackage{tcolorbox}
\usepackage{cleveref}

% Local commands go here.

%If you want glossaries
%\input{aglossary.tex}
%\makeglossaries

\title{Deep Learning Approach to Real-Bogus Classification for LSST Alert Production}

% Optional subtitle
% \setDocSubtitle{A subtitle}

\author{%
Nima Sedaghat
}

\setDocRef{DMTN-216}
\setDocUpstreamLocation{\url{https://github.com/lsst-dm/dmtn-216}}

\date{\vcsDate}

% Optional: name of the document's curator
% \setDocCurator{The Curator of this Document}

\setDocAbstract{%
This document provides precise definition(s) for the deep learning-based approach to the real-bogus classification task for LSST.
}

% Change history defined here.
% Order: oldest first.
% Fields: VERSION, DATE, DESCRIPTION, OWNER NAME.
% See LPM-51 for version number policy.
\setDocChangeRecord{%
  \addtohist{1}{YYYY-MM-DD}{Unreleased.}{Nima Sedaghat}
}


\begin{document}

% Create the title page.
\maketitle
% Frequently for a technote we do not want a title page  uncomment this to remove the title page and changelog.
% use \mkshorttitle to remove the extra pages

% ADD CONTENT HERE
% You can also use the \input command to include several content files.
\section{Problem Definition}

There are at least two approaches..
\\- traditional r/b classification
\\- modern (Transinet-style)\\

(see figure~\ref{fig:diagram})
each with its own pros and cons..

The two will be implemented and tested in parallel. Therefore we shall try to bring the two implementations close to eachother -- ideally unify.


\subsection{Basic Real/Bogus Classification}

\begin{itemize}
  \item input: cutouts from diff image with the potential transient at the center
  \item output: reliability score for the potential transient
\end{itemize}

Features:
\begin{itemize}
  \item can handle single object per cutout -- when there are multiple true transients, the behaviour is ..?
  \item the learning-based module does not have the chance to 
\end{itemize}

\paragraph{The score}
.. PSF-ness ..

\subsection{TransiNet: End-to-End Simultaneous Localization and Classification}

\begin{itemize}
  \item input: single calexp
  \item output: ``score image'': scores are assigned per-pixel
\end{itemize}


\paragraph{The score}
There are bogus detections, variables, ``appearing transients''
\begin{equation}
  s=\left| \frac{A_2-A_1}{A_2+A_1} \right|
\end{equation}


\begin{figure}
  \centering
  \includegraphics[width=1\textwidth]{material/diagram}
  \caption{Coarse illustration of information flow through the Alert Production (AP) pipeline. Paths in blue and green illustrate the ``traditional'' and ``modern'' approaches respectively.}
  \label{fig:diagram}
\end{figure}


% \section{Notes from the template}

% \subsection{How to handle LSST standard references?} 

% The papers should cite standard LSST references\footnote{See \url{https://github.com/lsst-pst/LSSTreferences}}, 
% where appropriate. For the usage, please see below.  These examples all use the ADS handle, unless they are 
% project docs then the use the project handle like LSE-17.

% All are on the lsst-texmf which you can get from \url{http://lsst-texmf.lsst.io}


% \subsubsection{LSST System and Science}

% The LSST system (brief overview of telescope, camera and data management subsystems),
% science drivers and science forecasts are described in:

% \begin{itemize}
% \item LSST Science Requirements Document: \cite{LPM-17}.
% \item LSST overview paper: \cite{2008arXiv0805.2366I}.
% \item LSST Science Book: \cite{abell2009lsst}.
% \end{itemize}
% %------------------------------------------------------------------------------


% \subsubsection{Simulations}

% The LSST simulations are described in a series of papers. Use of the LSST simulations should cite the LSST simulations overview paper \cite{2014SPIE.9150E..14C} and the specific simulation tools used:

% \begin{itemize}
% \item LSST Catalogs (CatSim): \cite{2014SPIE.9150E..14C}
% \item Feature-Based Scheduler: \cite{2018arXiv181004815N}
% \item Operations Simulator (OpSim): Scheduler \cite{2016SPIE.9910E..13D}, SOCS \cite{2016SPIE.9911E..25R}
% \item Metrics Analysis Framework (MAF): \cite{2014SPIE.9149E..0BJ}
% \item Image simulations (Phosim): \cite{2015ApJS..218...14P}
% \item Sky brightness model: \cite{2016SPIE.9910E..1AY}
% \item LSST Performance for NEO (or moving object) discovery: \cite{2018Icar..303..181J}
% \end{itemize}
% %------------------------------------------------------------------------------


% \subsubsection{Data Management}

% LSST data management system and the data products are described in:

% \begin{itemize}
%   \item The LSST Data Management System: \cite{2015arXiv151207914J}
%   \item Data Products Definition Document: \cite{LSE-163}
% \end{itemize}
%  %------------------------------------------------------------------------------


% \subsubsection{Camera}

% \begin{itemize}
%    \item Design and development of the LSST camera: \cite{2010SPIE.7735E..0JK}
% \end{itemize}
% %------------------------------------------------------------------------------


% \subsubsection{Telescope and Site}

% \begin{itemize}
%    \item Telescope and site overview and status in 2014:  \cite{2014SPIE.9145E..1AG}
% \end{itemize}
% %------------------------------------------------------------------------------

% \subsubsection{System Engineering}

% \begin{itemize}
%    \item LSST systems engineering: \cite{2014SPIE.9150E..0MC}
%    \item System verification and validation: \cite{2014SPIE.9150E..0NS}
% \end{itemize}
% %




\appendix
% Include all the relevant bib files.
% https://lsst-texmf.lsst.io/lsstdoc.html#bibliographies
\section{References} \label{sec:bib}
\renewcommand{\refname}{} % Suppress default Bibliography section
\bibliography{local,lsst,lsst-dm,refs_ads,refs,books}

% Make sure lsst-texmf/bin/generateAcronyms.py is in your path
\section{Acronyms} \label{sec:acronyms}
\addtocounter{table}{-1}
\begin{longtable}{p{0.145\textwidth}p{0.8\textwidth}}\hline
\textbf{Acronym} & \textbf{Description}  \\\hline

ADS & Astrophysics Data System \\\hline
DAQ & Data Acquisition System \\\hline
DM & Data Management \\\hline
EPO & Education and Public Outreach \\\hline
LPM & LSST Project Management (Document Handle) \\\hline
LSE & LSST Systems Engineering (Document Handle) \\\hline
LSST & Legacy Survey of Space and Time (formerly Large Synoptic Survey Telescope) \\\hline
MAF & Metric Analysis Framework \\\hline
NEO & Near-Earth Object \\\hline
OpSim & Operations Simulation \\\hline
\end{longtable}

% If you want glossary uncomment below -- comment out the two lines above
%\printglossaries





\end{document}
